\documentclass{beamer}
 
\input ../slidemac.tex

\title[NFPL242]{Automatizce ve fyzice: úvodní hodina}

\begin{document}
   
\maketitle


% ----------------------------------------------------------------------

\begin{markdown}
%%begin novalidate
### Úvodní informace

- Web cvičení: [cermak.science/teaching/automation/](https://cermak.science/teaching/automation/)
- Podmínky zápočtu: 
    - Překonání hard problému, nebo
    - Vyřešení 4 / 6 domácích úkolů
    \pause
- Konzultace: 
    - discord (nejrychlejší)
    - mail (klasika)  
    - osobní setkání (kancelář mám na Karlově)

\end{frame}

### Automatizace - proč?

- Zjednodušení práce - od kdy je to efektivní?
- Reprodukovatelnost
- Přenositelnost
- F.A.I.R. principy \pause

\end{frame}

### Programovací jazyk?
 
- Strojový kód je pro člověka nečitelný
- Závisí na procesoru
- Programovací jazyk je čitelný, *překladač* ho zkompiluje do zdrojového kódu
- Dynamické programovací jazyky - překládají se za běhu \pause

**Python**

- Snadno čitelný kód
- Populární ve vědě, AI, statistice
- Mnoho dostupných knihoven
- Triky, urychlení

**Excel**

- Nezatracujte ho - je to programování?
- Python binding!!

**Julia, R**

\end{frame}

### Git?
    
- Ukládání kódu
- Verzování
- https://gitlab.mff.cuni.cz/cermp5am/automation

### CI/CD ?

- Continuous Integration / Continuous Deployment
- Naměřím výsledky
- script je automaticky zpracuje
- gitlab!
    
\end{frame}

%%novalidate
\end{markdown}
% ----------------------------------------------------------------------

\begin{frame}{Software, cloud, [C,G,T]PU}

    \={Editor: VS Code.}
    
\end{frame}

% ----------------------------------------------------------------------

\begin{frame}{Co umíme? 1}

\py{1+1}{2}

\py{2+3*4+1}{15}

\py{2+3 * 4+1}{15} 

\py{(2+3)*(4+1)}{25}

\py{2**10}{1024}

\py{2**100}{1267650600228229401496703205376}

\end{frame}


% ----------------------------------------------------------------------

\begin{frame}{Co umíme? 2}
 
    \pycode{e1.py}
    
    \bigskip
    
    Výsledek je a+b=8
    
\end{frame}

% ----------------------------------------------------------------------

\begin{frame}{Co umíme? 3}
 
    \pycode{e2.py}
    
    \bigskip
    
    Průměr: 54.0
    
\end{frame}

% ----------------------------------------------------------------------

\begin{frame}{Co umíme? 4}
 
    \pycode{e3a.py}
    
\end{frame}

% ----------------------------------------------------------------------

\begin{frame}{Co umíme? 4}
 
    \pycode{e3b.py}
    
\end{frame}

% ----------------------------------------------------------------------

\begin{frame}{Co umíme? 4}
 
    \pycode{e3c.py}
    
\end{frame}
    

% ----------------------------------------------------------------------

% ----------------------------------------------------------------------

\begin{frame}{Navažujeme prvky}
 
 \pycode{first.py}
 
 \bigskip
 
 Znáte rychlejší způsob?
 
\end{frame}


\end{document}
